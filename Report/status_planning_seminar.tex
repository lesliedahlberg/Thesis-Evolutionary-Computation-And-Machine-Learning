%!TEX option = -shell-escape
% Document type
\documentclass[10pt, titlepage, a4paper]{article}



	% Packages


  \usepackage{graphicx}
	\usepackage{amsmath}
  \usepackage[utf8]{inputenc}
	\usepackage{amsfonts}
	\usepackage{fancyhdr}
	\usepackage{enumerate}
	\usepackage{listings}
    \usepackage{minted}
	\usepackage[titletoc]{appendix}
	\usepackage[pdfborder={0 0 0},colorlinks=true, urlcolor=blue, citecolor=red, bookmarks=false]{hyperref}
	\usepackage[margin=3cm]{geometry}
\usepackage[absolute]{textpos}
	\usepackage[section]{placeins}
	\usepackage{url}
  \usepackage{algorithm}
	\usepackage{tabularx}
%	\usepackage{gensymb}
	\usepackage{caption}
  \usepackage{subcaption}
  \usepackage{float}
  \usepackage{mathtools}
  \usepackage{amssymb}
  \usepackage{tikz}
  \usepackage{algpseudocode}
  \usepackage{listings}

%	\usepackage{xltxtra}
	% Page style
	\pagestyle{fancy}
	\marginparsep = 0pt
  \graphicspath{ {figures/} }



	% Set font
	%\setromanfont{Calibri}

	\renewcommand\contentsname{Contents}
	\newcommand{\HRule}{\rule{\linewidth}{0.5mm}}
	\renewcommand{\abstractname}{Abstract}


	\newcommand{\circR}{\textsuperscript{\textregistered}}

	% Code style
	\lstset{
		backgroundcolor=\color[rgb]{0.92,0.92,0.92},
		basicstyle=\footnotesize,
		showspaces=false,
		showstringspaces=false,
		showtabs=false,
		tabsize=2,
		captionpos=b,
		breaklines=true,
		keywordstyle=\color[rgb]{0,0,1},
		commentstyle=\color[rgb]{0.133,0.545,0.133}
	}

\begin{document}
\begin{titlepage}
% Title
%\title{

\begin{center}
		\begin{figure}[t]
				\includegraphics[width=15mm, bb=0 0 100 100]{MDHlogga.png}
		\end{figure}
                 	\Large M\"{a}lardalen University \\
			\Large School of Innovation Design and Engineering \\
                        \Large V\"{a}ster\r{a}s, Sweden\\

                        \noindent\makebox[\linewidth]{\rule{\textwidth}{0.4pt}}\\[0.5cm]

                %The complete name of the course you are enrolled in
                \Large{Thesis for the Degree of Bachelor in Computer Science 15.0 credits}\\[2.0cm]

			\huge \textbf{\uppercase{Emerging Evolutionary Algorithms for Continuous Optimization and Machine Learning}} \\ [2.5cm] %TITLE!!!!!!!!

			\LARGE Leslie Dahlberg   \\
        	\large ldg14001@student.mdh.se \\[2.5cm]

\begin{flushleft}
			\Large Examiner: \begin{minipage}[t]{0,7\textwidth}\Large \\\large M\"{a}lardalen University, \large V\"{a}ster\r{a}s, Sweden\\ \end{minipage}\\[0.5cm]
		%	\Large M\"{a}lardalen University\\
		%	\Large V\"{a}ster\r{a}s, Sweden\\[1.0cm]

			\Large Supervisor: \begin{minipage}[t]{0,7\textwidth}\Large Ning Xiong\\\large M\"{a}lardalen University, \large V\"{a}ster\r{a}s, Sweden \end{minipage} \\[0.5cm]
		%	\Large M\"{a}lardalen University\\
		%	\Large V\"{a}ster\r{a}s, Sweden\\[0.5cm]

                      %  \Large Company supervisor: \begin{minipage}[t]{0,6\textwidth}\Large Supervisor Name, \\\large Company name, location \end{minipage}
		%	\Large Company\_Name\\
		%	\Large Location
\end{flushleft}

              \vspace*{\fill}
                    \large \today		%today should be replaced by the date the report is sent for examination

\end{center}

\end{titlepage}
%\date{}
%\maketitle

% Page style
\thispagestyle{fancy}
\fancyhead[R]{Leslie Dahlberg}
\fancyhead[L]{M\"{a}lardalen University}
\fancyfoot[L]{}
%\fancyfoot[LE,RO]{\thepage}
\renewcommand{\headrulewidth}{0.4pt}
\renewcommand{\footrulewidth}{0.4pt}

% Begin actual text

\def\abstract{
   \vfil
\begin{center}%
{\bfseries\abstractname\vspace{-.5em}}
\end{center}
\itshape
}

\def\endabstract{\par
}


\newpage







% ========================== Document version ===========================
%\begin{center}
%	\begin{tabular}{|l|l|l|}
 %
	%	\multicolumn{3}{c}{\textbf{{\large Document version}}} \\
	%	\hline
 	%	Version & Date & Note \\ \hline
 	%	 &  &  \\ \hline
	%	 &  &  \\ \hline
	%	 &  &  \\ \hline
	%	 &  &   \\ \hline
	%	 &  &   \\ \hline
	%	 &  &   \\ \hline
%	\end{tabular}
%\end{center}
%\newpage
%========================== Table of contents ===========================
\hypersetup{linkcolor=black}
\tableofcontents
\hypersetup{linkcolor=red}

\newpage

% ============================== Intro ===============================

\input{./introduction}
\newpage
\input{./background}
\newpage
\section{Related Work}

Ideas around evolutionary computation began emerging in 1950s. Several researchers, independently from each-other, created algorithms which were inspired by natural Darwinian principles, these include Holland's Genetic Algorithms, Schwefel's and Rechenberg's Evolution Strategies and Fogel's Evolutionary Programming. These pioneering algorithms shared the concepts of populations, individuals, offspring and fitness and ,compared to natural systems, they were quite simplistic, lacking gender, maturation processees, migration, etc \cite{dejong2009EC}.

Research has shown that no single algorithm can perform better than all other algorithms on average. This has been referred to as the `no-free-lunch' and current solutions instead aim at finding better solutions to specific problems by exploiting inherent biases in the problem. This has led to the desire to classify different algorithms in order to decide which algorithms should be used in which situations, a problem which is not trivial \cite{dejong2009EC}.

Recent research has focused, among others, on parallelism, multi-population models, multi-objective optimization, dynamic environments and evolving executable code. Parallelism can easily be exploited in EC because of it's inherently parallel nature, e.g each individual in a population can be evaluated, mutated and crossed-over independently. Multi-core CPUs, massively parallel GPUs, clusters and networks can be used to achieve this. Multi-population models mimic the way species depend on each other in nature. Examples of this include host-parasite and predator-prey relationships where the the individual's fitness is connected to the fitness of another individual. Multi-objective optimization aims to solve problems where conflicting interests exist, a good example would be optimizing for power and fuel-consumption simultaneously. In such problems the optimization algorithm has to keep two or more interests in mind simultaneously and find intersections points which offer the best trade-offs between them. Dynamic environments include things like the stock markets and traffic systems. Traditional EAs perform badly in these situations but they can perform well when slightly modified to fit the task. Evolving executable code, as in Genetic Programming and Evolutionary Programming, is a hard problem with very interesting potential applications. Most often low-level code such as assembly, lisp or generic rules are evolved \cite{dejong2009EC}.

\subsection{Differential Evolution}

Differential evolution (DE) was conceived in 1995 by Storn and Price \cite{storn1995differential} and soon gained wide acceptance as one of the best algorithms in continuous optimization \cite{price1997differential}. This spawned many new papers descibing variations and hybrids of the algorithm \cite{5601760}, such as self-adaptive DE (SaDE) \cite{qin2009differential}, opposition-based DE (ODE) \cite{rahnamayan2008opposition} and DE with global and local neighborhoods (DEGL) \cite{rahnamayan2008opposition}.

DE is very easy to implement and has been shown to outperform most other algorithms consistently, it has also been shown that it in general performs better than PSO \cite{das2009differential, vesterstrom2004comparative}. DE uses very few patameters and the effects of altering these parameters have been well studied \cite{5601760}. DE comes in a total of 10 different varieties based on which mutation and cross-over operators are used \cite{price2006differential}. Eight of these schemes have been tested and compared, showing that the version called DE/best/1/bin (which utilizes the best current individual in the cross-over process instead of random invidivuals) generally yields the best results \cite{mezura2006comparative}. \cite{gamperle2002parameter} measured the performance of different combinations of parameters, producing general recommendations for DE.

The desire to find optimal parameters have led to the use of self-adjusting DE algorithms. Examples inlude the use of fuzzy systems to control the parameters \cite{liu2005fuzzy} and the SaDE algorithm \cite{qin2009differential}.

\subsection{Particle Swarm Optimization}

Particle swarm optimization (PSO) is the most widely used swarm intelligence (SI) algorithm to date. Many modified versions of PSO have been proposed, among others quantum-behaved PSO (QPSO), bare-bones PSO (BBPSO), chaotic PSO, fuzzy PSO, PSOT-VAC and opposition-based PSO. PSO has also been hybridized with other evolutionary algorithm, for instance: genetic algorithms (GA), artificial immune systems (AIS), tabu search (TS), ant colony optimization (ACO), simulated annealing (SA), differential evolution (DE), bio-geography based optimization (BBO) and harmonic search (HS) \cite{zhang2015comprehensive}.

Wang et al. \cite{dang2012selection} have compared the performance of different PSO-parameters and proposed a set of criteria for improving the performance of PSO. Fuzzy logic controllers (FLC) have been used to continuously optimize the PSO-parameters by Kumar and Chaturvedi \cite{kumar2011tuning}. Zhang et al. found a simple way to use control theory in order to find good parameters \cite{zhang2011simple}. Yang proposed modified velocity PSO (MVPSO) in which particles learn the best parameters from the other particles \cite{yang2011particle}.


\subsection{Estimation of Distribution Algorithms}

\subsection{Evolutionary Algorithms in Machine Learning}




{\color{blue}
\begin{itemize}
  \item Syftet med detta avsnitt \"ar att placera in ditt arbete i ett sammanhang och j\"amf\"ora det med tidigare publicerade arbeten och resultat inom omr\r{a}det. Denna del ska vara grundlig. Du beskriver h\"ar existerande kunskap och hur denna ut\"okas av ditt arbete. Den ska inneh\r{a}lla analyser av tidigare arbeten som exempelvis beskriver hur olika metoder skiljer sig \r{a}t. Du ska visa p\r{a} de viktigaste likheterna och skillnaderna betr\"affande uppgift, angreppss\"att/metodologi samt resultat. Det \"ar viktigt att du p\r{a} ett neutralt s\"att diskuterar f\"or- och nackdelar med ditt eget arbete j\"amf\"ort med andras.
  \item Detta skapar ocks\r{a} en f\"orv\"antan p\r{a} bidraget f\"or ditt arbete, l\"asaren l\"ar sig h\"ar om begr\"ansningar hos tidigare arbeten och varf\"or din uppgift \"ar en utmaning..
  \item Tillsammans kommer detta avsnitt tillsammans med bakgrund att introducera ``state of the art''/``state of practice'' och dess brister, betydelsen av uppgiften samt vad ditt arbete ska j\"amf\"oras med.
\end{itemize}




}

\newpage
\section{Problem Formulation}

The purpose of this thesis is to explore the performance and usefulness of three emerging evolutionary algorithms: differential evolution (DE), particle swarm optimization (PSO) and estimation of distribution algorithms (EDA). The intention is to compare these against each other on a set of benchmark functions and practical problems in machine learning and then, if possible, develop a new or modified algorithm which improves upon the aforementioned ones in some aspect.

\paragraph{Research Questions}

The questions asked in the thesis are:
\begin{itemize}
  \item How do DE, PSO and EDA perform comparatively when applied to mathematical optimization problems?
  \item How do DE, PSO and EDA perform comparatively when applied to machine learning problems such as neural network optimization and artificial intelligence in games?
  \item How are DE, PSO and EDA suited to these different problems?
  \item Can an improved algorithm which draws inspiration from the design of DE, PSO and/or EDA outperform any of them in some/all of the aforementioned benchmarks and problem sets?
\end{itemize}

\paragraph{Motivation}

This research is interesting because it yields insight into the applicability of emerging evolutionary algorithms to currently popular machine learning methods such as artificial neural networks and also compares them more generally on generic mathematical optimization problems. The possibility of an improved novel algorithm which is better at handling machine learning problems also makes the work more interesting.

\paragraph{Outcomes}

The goals in this work are:
\begin{itemize}
  \item Benchmark DE, PSO, EDA on mathematical optimization problems
  \item Benchmark DE, PSO, EDA on machine learning problems
  \item Develop a new algorithm inspired by DE, PSO and/or EDA
  \item Benchmark the new algorithm
  \item Compare the new algorithm with DE, PSO and EDA and draw conclusions from the results
\end{itemize}

\paragraph{Limitations}

The scope of this work limits the number of algorithms which can be included in the testing. The individual algorithms also have numerous variations and parameters which can dramatically affect their behavior and it will not be possible to take all these variations into considerations. Furthermore, the benchmarking will be restricted to a standard set of testing functions which may or may not provide reliable information regarding the general usability of the algorithms. Since evolutionary algorithms have a large number of potential and actual use-cases, the practical testing will only concern a small subset of these and may therefore not provide accurate data for all possible use-cases.

\newpage

\section{Time plan}

The thesis work planing will be based upon the structure of the written report to ensure that the final product is delivered in time. The first two weeks will serve as an introduction to the topic and the introduction and background section of the report will be completed during this period. This will guarantee that they will be available in time for the status and planning seminary. After this, deeper inquiry into the main algorithms of the thesis will be conducted and the first implementation prototypes will be constructed. During this time a benchmark methodology will also be decided. This phase might occupy 1 - 3 weeks and will be followed by a benchmark and an investigation of how to apply the algorithms to machine learning problems. This will probably occupy one week and will be followed by the development of an improved algorithm and the implementation of the machine learning benchmarks. After this a complete benchmark of all four algorithms on both standard functions and machine learning problems will be conducted. This should leave enough time over to fine tune the new algorithm and finalize the report before the end of the course/project period.

\newpage

% ============================= References ============================

%\textbf{Referenser (References)}

F\"or examensarbeten ska du ha vetenskapliga referenser. En vetenskaplig referens \"ar en referens till en artikel som publicerats med syfte att sprida kunskap som resultat av metodiskt rigor\"ost driven forskning kring en forskningsfr\r{a}ga. Artikeln har granskats kollegialt genom s.k peer review och publicerats i en tidskrift eller s.k. konferens-proceedings. En vetenskaplig referens har grundl\"aggande information f\"or att identifiera artikeln: f\"orfattare, \r{a}rtal, titel, namn p\r{a} konferens eller tidskrift med volym, nummer och sidor samt f\"orlag. F\"or vetenskapliga referenser ska du inte ange DOI, URL eller n\"ar du h\"amtade den. Alla andra referenser har ett mindre vetenskapligt v\"arde: l\"arob\"ocker, studentuppsatser, journalistiska tidskrifter, branschtidskrifter, standards, popul\"arvetenskapliga publikationer, wikipedia, uppslagsverk, web-l\"ankar, TV- och radioprogram, bloggar och inl\"agg, alla dessa \"ar s\r{a}ledes inte vetenskapliga referenser. De kan anv\"andas som referenser men betraktas som komplement till de vetenskapliga referenserna.

Referenser ska anges enligt IEEE-standarden. I den l\"opande texten anger du referenserna med siffror (l\"opnummer) i hakparentes, t ex [1] i den ordning de f\"orekommer f\"orsta g\r{a}ngen i texten. F\"orekommer en referens flera g\r{a}nger i texten ska den anges med samma nummer varje g\r{a}ng. Om du vill ange ett kapitel eller en eller flera specifika sidor i referensen s\r{a} skriver du det efter l\"opnumret inom hakparentesen. T ex om du har ett citat ur en text s\r{a} anges detta med sidnummer, ex [2 s. 3]
N\"ar du s\"atter ut referenser i texten, s\r{a} t\"ank p\r{a} att de ska placeras i n\"ara anslutning till det som referensen handlar om, och alltid inom en mening, dvs f\"ore ’.’ Undantag fr\r{a}n detta \"ar om ett helt stycke behandlar samma referens, d\r{a} kan referensen s\"attas efter styckets slut.

Sist i rapporten ska det finnas en referenslista. I den ska finnas samtliga referenser du \r{a}beropar i texten, och du m\r{a}ste ocks\r{a} se till att du i texten verkligen refererar till samtliga i listan. L\"as om hur du anger k\"allorna i referenslistan p\r{a} \url{http://www.ieee.org/documents/ieeecitationref.pdf} alternativt i manualen \url{https://www.ieee.org/documents/style_manual.pdf}. Du kan ocks\r{a} g\r{a} via bibliotekets lista \"over guider till olika referensstilar som du hittar p\r{a} \url{http://www.mdh.se/bibliotek/kurser-stod/referera/guider-till-olika-referensstilar-1.79882?l=sv_SE}

Det \"ar mycket viktigt att referenshanteringen \"ar korrekt, b\r{a}de inne i texten och i sj\"alva listan! Detta \"ar en sak vi alltid tittar p\r{a} n\"ar vi bed\"omer rapporten. 

Det finns ocks\r{a} verktyg som hj\"alper dig att hantera referenser r\"att, Endnote (f\"or Word), se \url{http://www.mdh.se/bibliotek/kurser-stod/referera/endnote/} eller BibTex (f\"or LaTeX), se \url{http://www.bibtex.org/}.






\bibliographystyle{IEEEtran}
\bibliography{./references}
\addcontentsline{toc}{section}{References}

% ============================ Appendices =============================
\newpage
%\begin{appendices}



	%\section{Algorithms}
\label{appendix:algorithms}

This appendix will present the matlab code for the proposed algorithm.

\subsection{Differential Estimation of Distribution}

\lstinputlisting[language=matlab]{../Code/algorithms/DEDA.m}

	%\clearpage

%\end{appendices}

\end{document}
