\textbf{Referenser (References)}

F\"or examensarbeten ska du ha vetenskapliga referenser. En vetenskaplig referens \"ar en referens till en artikel som publicerats med syfte att sprida kunskap som resultat av metodiskt rigor\"ost driven forskning kring en forskningsfr\r{a}ga. Artikeln har granskats kollegialt genom s.k peer review och publicerats i en tidskrift eller s.k. konferens-proceedings. En vetenskaplig referens har grundl\"aggande information f\"or att identifiera artikeln: f\"orfattare, \r{a}rtal, titel, namn p\r{a} konferens eller tidskrift med volym, nummer och sidor samt f\"orlag. F\"or vetenskapliga referenser ska du inte ange DOI, URL eller n\"ar du h\"amtade den. Alla andra referenser har ett mindre vetenskapligt v\"arde: l\"arob\"ocker, studentuppsatser, journalistiska tidskrifter, branschtidskrifter, standards, popul\"arvetenskapliga publikationer, wikipedia, uppslagsverk, web-l\"ankar, TV- och radioprogram, bloggar och inl\"agg, alla dessa \"ar s\r{a}ledes inte vetenskapliga referenser. De kan anv\"andas som referenser men betraktas som komplement till de vetenskapliga referenserna.

Referenser ska anges enligt IEEE-standarden. I den l\"opande texten anger du referenserna med siffror (l\"opnummer) i hakparentes, t ex [1] i den ordning de f\"orekommer f\"orsta g\r{a}ngen i texten. F\"orekommer en referens flera g\r{a}nger i texten ska den anges med samma nummer varje g\r{a}ng. Om du vill ange ett kapitel eller en eller flera specifika sidor i referensen s\r{a} skriver du det efter l\"opnumret inom hakparentesen. T ex om du har ett citat ur en text s\r{a} anges detta med sidnummer, ex [2 s. 3]
N\"ar du s\"atter ut referenser i texten, s\r{a} t\"ank p\r{a} att de ska placeras i n\"ara anslutning till det som referensen handlar om, och alltid inom en mening, dvs f\"ore ’.’ Undantag fr\r{a}n detta \"ar om ett helt stycke behandlar samma referens, d\r{a} kan referensen s\"attas efter styckets slut.

Sist i rapporten ska det finnas en referenslista. I den ska finnas samtliga referenser du \r{a}beropar i texten, och du m\r{a}ste ocks\r{a} se till att du i texten verkligen refererar till samtliga i listan. L\"as om hur du anger k\"allorna i referenslistan p\r{a} \url{http://www.ieee.org/documents/ieeecitationref.pdf} alternativt i manualen \url{https://www.ieee.org/documents/style_manual.pdf}. Du kan ocks\r{a} g\r{a} via bibliotekets lista \"over guider till olika referensstilar som du hittar p\r{a} \url{http://www.mdh.se/bibliotek/kurser-stod/referera/guider-till-olika-referensstilar-1.79882?l=sv_SE}

Det \"ar mycket viktigt att referenshanteringen \"ar korrekt, b\r{a}de inne i texten och i sj\"alva listan! Detta \"ar en sak vi alltid tittar p\r{a} n\"ar vi bed\"omer rapporten. 

Det finns ocks\r{a} verktyg som hj\"alper dig att hantera referenser r\"att, Endnote (f\"or Word), se \url{http://www.mdh.se/bibliotek/kurser-stod/referera/endnote/} eller BibTex (f\"or LaTeX), se \url{http://www.bibtex.org/}.




