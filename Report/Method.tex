\subsection{Method}

The thesis has a dual contribution, a benchmark of existing algorithms on a variety of problems and the development of a new approach followed by a benchmark to compare it to the existing algorithms.

To ensure the reliability of the results in the benchmark a statistical approach where multiple measurements are perform is used. The final recorder results are the mean value of the measurement and the standard deviation. This is also the approach used by Das et al.~\cite{Das2008} when comparing evolutionary algorithms and their hybridizations. The number of repeat measurements has been chosen to be sufficiently high but is also constrained by the processing time required to do the computation. The neural network benchmarks suffer most from this and the number of repeat measurement has therefore been lower in their case compared to the mathematical benchmark. The details of the measurement methodology is presented later in the text together with the problem sets themeselves.

The development of the new algorithm followed an experimental approach where different completely new ideas were tried and mixed with hybridization prototypes. The algorithms were continuously tested an compared to the benchmark of the standard algorithms until a sufficiently good proptype was developed.

The algorithms, the benchmark set-up, the mathematical functions, the neural networks and the game controllers used in this work were all developed and tested in Matlab. The choice of Matlab was made due to the availability of useful mathematical tools and tool-boxed, such as easy matrix manipulation and probability distributions, which made the development of the algorithms and benchmarks faster and more convenient.
